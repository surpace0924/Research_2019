\documentclass[autodetect-engine,dvipdfmx-if-dvi,ja=standard, 12pt]{bxjsarticle}

% 二段組にするとき
% \documentclass[twocolumn,autodetect-engine,dvipdfmx-if-dvi,ja=standard]{bxjsarticle}

\usepackage{graphicx}        %図を表示するのに必要
\usepackage{color}           %jpgなどを表示するのに必要
\usepackage{amsmath,amssymb} %数学記号を出すのに必要
\usepackage{setspace}
\usepackage{cases}
\usepackage{here}
\usepackage{fancyhdr}
\usepackage{ascmac}

\setlength{\textheight}{\paperheight}   % 紙面縦幅を本文領域にする(BOTTOM=-TOP)
\setlength{\topmargin}{3truemm}       % 上の余白を30mm(=1inch+4.6mm)に
\addtolength{\topmargin}{-\headheight}  %
\addtolength{\topmargin}{-\headsep}     % ヘッダの分だけ本文領域を移動させる
\addtolength{\textheight}{-50truemm}    % 下の余白も30mm(BOTTOM=-TOPだから+TOP+30mm)
% #################### Landscape Setting #######################
% # LEFT = 1inch + \hoffset + \oddsidemargin (\evensidemargin) #
% #      = 1inch + 0pt + 0pt                                   #
% # RIGHT = \paperwidth - LEFT - \textwidth                    #
% ##############################################################
\setlength{\textwidth}{\paperwidth}     % 紙面横幅を本文領域にする(RIGHT=-LEFT)
\setlength{\oddsidemargin}{-5.4truemm}  % 左の余白を25mm(=1inch-0.4mm)に
\setlength{\evensidemargin}{-5.4truemm} %
\addtolength{\textwidth}{-40truemm}     % 右の余白も25mm(RIGHT=-LEFT)

% 行頭の字下げをしない
\parindent = 0pt

% ヘッダとフッタの設定
\lhead{工業英語}
\chead{歩行者位置予測による運転抵抗を考慮した高度減速制御}
\rhead{5E 20番 佐藤凌雅}
\lfoot{}
\cfoot{-\thepage-} % ページ数
\rfoot{}

% 式の番号を(senction_num.num)のようにする
\makeatletter
\@addtoreset{equation}{section}
\def\theequation{\thesection.\arabic{equation}}
\makeatother

% 画像の貼り付けを簡単にする
\newcommand{\pic}[2]
{
  \begin{figure}[H]
    \begin{center}
      \includegraphics[scale=#2]{#1}
    \end{center}
  \end{figure}
}

% 単位の記述を簡単にする
\newcommand{\unit}[1]
{
  \, [\mathrm{#1}]
}
\begin{document}

\begin{abstract}
 本研究は、車両と歩行者との衝突を回避するための自動減速システムの実現を目的としている。提案システムは現在位置から歩行者の将来位置を予測し、衝突確率を検出することができる。さらに、駆動抵抗やモデリング誤差を補償することができるモデル予測制御を採用したコントローラも提案されている。提案した方法の有効性をシミュレーションと実験を通して検証した。
\end{abstract}

\maketitle
\pagestyle{fancy}

\section{前書き}
 車両制御技術は、情報処理の先端技術として急速に発展している。たとえば、急ブレーキ時にホイールロックを防止するアンチロックブレーキシステム(ABS)、滑りやすい路面でタイヤがスリップするのを抑制する電子安定性制御(ESC)(1)。また、各車輪の駆動力を独立して制御して車両の挙動を安定させるダイレクトヨーモーメント制御(DYC)と、駆動力を制御することで所望のスリップ率を維持するスリップ率制御(2)(3)。このような制御技術の発展により、自動車はこれまで以上に安全かつ安定して走行することが可能になってきた。運転支援システムは、交通事故件数の削減に大きく貢献し、さらなる発展が期待されている。\\
 近年、交通事故を減らす技術として自律型緊急ブレーキシステム(AEBS)が注目されている。AEBSの概要:車両に搭載されたカメラやレーダーが他の車両や前方の歩行者などの障害物を検出した場合、相対位置と相対速度から衝突の危険性を判断する。すると,システムは車両を減速させる。自動車製造業者の調査によると、事故の合計数は、非装備車と比較して61%減少し、後端事故の数は84%減少した(4)。それはAEBSの重要性と有用性を証明している。\\
 しかしながら、現在の商用車のAEBSは、歩行者の現在位置に基づいて衝突の危険性を検出しており、歩行者の動きを考慮していないという問題がある。歩行者や障害物が車両の前方に存在する場合、システムはそれに対処することができ、衝突事故を回避することが可能である。一方、歩行者が横から歩くと、衝突の瞬間まで歩行者が車両の前に位置していないため、現在のAEBSは機能しない可能性がある。歩行者と車両との衝突事故は道路を横断する間にしばしば起こるので、歩行者運動に対応するAEBSは交通事故を減らすための重要なシステムである。\\
 歩行者の動きに対応して、歩行者の将来の位置を予測し、それに応じて制動力を制御する必要がある。したがって、有限時間における車両の位置がシステムの指令値となるように、時間領域で車両を制御する必要がある。\\
 そこで,時間領域での制御器設計法としてモデル予測制御(MPC)が注目されている。MPCは制約条件付きリアルタイム最適制御と呼ぶことができ、将来のシステム挙動を予測し、評価関数 (5) を最小化するように現在の入力値を決定する。MPCは二次計画問題を解決する必要があり,計算負荷が大きいため,プロセス制御の分野のようにサンプリング時間の長いシステムで広く使用されている。しかし,近年,計算機の計算速度が向上してきており,サンプリング時間の短いシステムへの適用が可能になってきている。MPCのもう一つの問題は,開ループコントローラである。制御性能はモデル精度に大きく依存し,外乱の感度が高いという欠点がある。\\
 外乱抑制と目標値への追従性を両立させるためには、外乱を推定しフィードフォワード制御で補償する必要がある。外乱とは、転がり抵抗、空気抵抗、坂道走行時に発生する勾配抵抗などの走行抵抗のことである。しかしながら、駆動力制御に関するこれまでの研究のほとんどは、駆動抵抗が非常に小さいと仮定し無視している(6)。サンプリング時間が短いシステムの場合、フィードバックループによる駆動抵抗の影響を少なくすることができる。しかしながら、MPCはPID制御等よりもサンプリング時間が長く、外乱の影響が顕著になる。さらに、外乱はMPCにおける予測に使用されるモデルの精度を低下させるので、自動車の予測された挙動は実際のものから逸脱する。これでは所望の結果が得られない。したがって、MPCを駆動力制御に適用するためには、駆動抵抗の推定と補償を考慮する必要がある。\\
 一般に、人間はさまざまな道路環境や歩行者に基づいて危険を予測しながら運転する。ここでは、人間などの危険性を予測しながら車両を制御することを目的としており、歩行者の動きに対応するためのAEBSの改良につながる。この目的を達成するために、歩行者の将来の位置を予測するためのアルゴリズムの構築と運転抵抗を補償することができる時間領域におけるコントローラの設計について述べる。\\
 本論文では、車両に搭載されたカメラなどのセンシング機器から得られる歩行者の位置情報を想定している。提案手法の概要は以下の通りである。まず、歩行者速度は、現在の歩行者の位置からカルマンフィルタによって推定される。将来の歩行者の予測位置は、一定期間の歩行者速度の値から予測される。衝突の可能性がある場合は、車両の予測位置と比較してMPCにより制動力の指令値を算出する。 MPCはモデルの精度に依存するため、外乱とパラメータ誤差の影響を強く受けるという欠点がある。そこで、制御性能を向上させるために制動力のフィードバックループを導入し、外乱である駆動抵抗の推定量を提案システムに含める。システムのコントローラはMPCと制動力フィードバック制御のカスケードコントローラである。提案した方法を利用することにより、歩行者の将来の位置を予測し、衝突の危険性を検出することが可能である。さらに、駆動抵抗を補償して制動力を制御することで、外乱感度を低下させながらシステムの追従性を向上させることができる。\\
 この論文は以下のように構成されている。 2章では提案手法のモデル化について説明し、3章で制御系の設計について述べる.4章と5章ではそれぞれシミュレーションと実験について説明する。最後に、結論は6章に示す。\\

\section{モデリング}
\subsection{車体モデル}
 車両と歩行者との衝突はまっすぐな道路上で起こることがほとんどであり、モデルを単純化するために直線運動のみを考慮した車両の車体モデルを図1に示す。\\
% \begin{figure}[]
%     \centering
%     \includegraphics[height=5cm]{./fig/1.png}
%     \caption{車両の車体モデル}
%   \end{figure}
 図1に示すように、車両の車体の運動学とダイナミクスは次のとおり。\\


 ここで $x$ と $y$ は車両の位置である. $v$、$\theta$、$M$、$n$、$F_d$、$F_r$ は、それぞれ車速、車両角度、車両質量、制御輪数、駆動力、駆動抵抗である。$\Sigma W$ はワールド座標系を表す。したがって、$\Sigma W$ の $x$ と $y$ は、それぞれ緯度と経度の位置を示す。一般的に、抵抗力は移動方向に対して反対方向を向いている。ただし、駆動抵抗は正の値でも負の値でも構わない。この論文では、$F_r$ は $F_d$ と同じ方向を向いている。\\
 ちなみに、本論文の制御法にはモデル予測制御(MPC)を含んでおり、MPCは離散状態で定式化される。したがって、車体モデルの線形移動のみを制限した離散時間状態空間表現は、次のように表される.\\



ここで $T_s$ はサンプリング時間を表している。 $u[k]$ は各車輪の力の合計であり、システムの出力変数は状態変数に等しい。 これは, $x$、$y$、$v$に関する情報を取得できることを示している。日本では準天頂衛星システム(QZSS)などの測位技術の開発により、車両の位置を高精度に計測できるため、論文はこれらの値が得られると仮定している(7)。

\subsection{ホイールモデル}

\end{document}
